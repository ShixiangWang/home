\documentclass[10pt,]{article}
\usepackage{lmodern}
\usepackage{amssymb,amsmath}
\usepackage{ifxetex,ifluatex}
\usepackage{fixltx2e} % provides \textsubscript
\ifnum 0\ifxetex 1\fi\ifluatex 1\fi=0 % if pdftex
  \usepackage[T1]{fontenc}
  \usepackage[utf8]{inputenc}
\else % if luatex or xelatex
  \usepackage{unicode-math}
  \defaultfontfeatures{Ligatures=TeX,Scale=MatchLowercase}
\fi
% use upquote if available, for straight quotes in verbatim environments
\IfFileExists{upquote.sty}{\usepackage{upquote}}{}
% use microtype if available
\IfFileExists{microtype.sty}{%
\usepackage[]{microtype}
\UseMicrotypeSet[protrusion]{basicmath} % disable protrusion for tt fonts
}{}
\PassOptionsToPackage{hyphens}{url} % url is loaded by hyperref
\usepackage[unicode=true]{hyperref}
\PassOptionsToPackage{usenames,dvipsnames}{color} % color is loaded by hyperref
\hypersetup{
            pdftitle={Curriculum Vitae},
            colorlinks=true,
            linkcolor=Maroon,
            citecolor=Blue,
            urlcolor=blue,
            breaklinks=true}
\urlstyle{same}  % don't use monospace font for urls
\usepackage[margin=1in]{geometry}
\IfFileExists{parskip.sty}{%
\usepackage{parskip}
}{% else
\setlength{\parindent}{0pt}
\setlength{\parskip}{6pt plus 2pt minus 1pt}
}
\setlength{\emergencystretch}{3em}  % prevent overfull lines
\providecommand{\tightlist}{%
  \setlength{\itemsep}{0pt}\setlength{\parskip}{0pt}}
\setcounter{secnumdepth}{0}
% Redefines (sub)paragraphs to behave more like sections
\ifx\paragraph\undefined\else
\let\oldparagraph\paragraph
\renewcommand{\paragraph}[1]{\oldparagraph{#1}\mbox{}}
\fi
\ifx\subparagraph\undefined\else
\let\oldsubparagraph\subparagraph
\renewcommand{\subparagraph}[1]{\oldsubparagraph{#1}\mbox{}}
\fi

% set default figure placement to htbp
\makeatletter
\def\fps@figure{htbp}
\makeatother

\usepackage[fontsize=13pt]{scrextend}
\usepackage{ctex}
\usepackage{fontspec}
%\usepackage{fontawesome}
\setmainfont{Palatino} % English font: Optima, FF Nexus Serif, Palatino
\setCJKmainfont{STKaiti} % Chinese serif font: Source Han Serif SC
\usepackage{hyperref}
%% 2. Change page style
\usepackage{datetime2}
\usepackage{fancyhdr}
\usepackage{lastpage} % LastPage
\pagestyle{fancy}
\renewcommand{\headrulewidth}{0pt}
\lhead{}
\rhead{Shixiang Wang's CV}
\rfoot{Last update: \today}
\fancyfoot[C]{--~\thepage/\pageref*{LastPage}~--}

\title{Curriculum Vitae}
\date{2016-01-22}

\begin{document}
%\maketitle

\hypertarget{likan-zhan}{%
\section{Likan Zhan}\label{likan-zhan}}

\begin{itemize}
\tightlist
\item
  Add: 15, Xueyuan Rd., Haidian District, Beijing 100083, China
\item
  Tel: +86 10 8230 3468
\item
  Email: zhanlikan AT hotmail DOT com
\item
  Personal website: \url{https://likan.info}
\end{itemize}

\hypertarget{academic-appointment}{%
\subsection{Academic appointment}\label{academic-appointment}}

\begin{itemize}
\tightlist
\item
  2014.10 \textasciitilde{} Now, Assistant professor,\\
  MEG Laboratory for Brain Sciences, \\
  Institute for Speech Pathology and the Brain Science,\\
  Beijing Language and Culture University, Beijing, China
\end{itemize}

\hypertarget{education}{%
\subsection{Education}\label{education}}

\begin{itemize}
\item
  2010.10 \textasciitilde{} 2014.09, Ph.D.~Cognitive Science, \\
  Macquarie University, Sydney, Australia
\item
  2007.09 \textasciitilde{} 2010.07, M.E. Cognitive Psychology, \\
  Beijing Language and Culture University, Beijing, China
\item
  2000.09 \textasciitilde{} 2004.07, B.A. Teaching Chinese as a Second
  Language, \\
  Beijing Language and Culture University, Beijing, China
\end{itemize}

\hypertarget{teaching-responsibilities}{%
\subsection{Teaching responsibilities}\label{teaching-responsibilities}}

\begin{itemize}
\item
  Statistics for the Behavioral Sciences,
  \href{https://likan.info/en/teach/Statistics-for-the-Behavioral-Sciences/}{More
  Info \ldots{}}
\item
  R for Modeling and Visualizing Data,
  \href{https://likan.info/en/teach/R-for-Statistics-Data-Visualization/}{More
  Info \ldots{}}
\item
  Introduction to Cognitive Neuroscience,
  \href{https://likan.info/en/teach/Introduction-to-Cognitive-Neuroscience/}{More
  Info \ldots{}}
\item
  Foundations of Scientific Research,
  \href{https://likan.info/en/teach/Science-and-Scientific-Research/}{More
  Info \ldots{}}
\end{itemize}

\hypertarget{professional-skills}{%
\subsection{Professional skills}\label{professional-skills}}

\begin{itemize}
\item
  Statistics. I teached two statistical courses to master students. I
  have moderate experience in using \emph{R} for statistical modeling
  and data visualization. Besides the basic R, I used the following R
  packages a lot: \href{http://r-datatable.com}{\emph{data.table}},
  \href{http://dplyr.tidyverse.org}{\emph{dplyr}},
  \href{http://ggplot2.tidyverse.org}{\emph{ggplot2}},
  \href{https://cran.r-project.org/web/packages/car/index.html}{\emph{car}},
  \href{https://github.com/lme4/lme4}{\emph{lme4}},
  \href{https://cran.r-project.org/web/packages/gam/index.html}{\emph{gam}}
  etc. I also created a R package myself, called
  \href{https://github.com/likanzhan/acqr}{\emph{acqr}}.
\item
  Experimental techniques. I'm familiar with
  \href{https://www.pstnet.com/eprime.cfm}{E-prime},
  \href{http://www.psychopy.org}{Psychopy},
  \href{http://psychtoolbox.org}{Psychtoolbox}, and
  \href{https://www.neurobs.com/presentation}{Presentation} for
  presenting test stimuli. I have advanced experience in using Eyelink
  II/1000 plus (SR Research Ltd.), such as \emph{Experiment Builder} for
  experiment building, and \emph{Data Viewer} for data analysis. I am in
  charge of establishing the first Child MEG lab in China, familiar with
  \emph{Yokogawa} MEG system for equipment maintenance and MEG data
  acquisition. I'm also familiar with \emph{BESA}, and some Matlab
  packages for analyzing the E/MEG data, such as
  \href{http://www.fieldtriptoolbox.org}{\emph{fieldtrip}} and
  \href{https://github.com/neurodebian/spm12}{\emph{SPM12}}.
\item
  Writing. I like to write with LaTeX, Markdown, and R Markdown. I
  created my \href{https://likan.info}{personal website} with
  \href{https://gohugo.io}{\emph{hugo}} and
  \href{https://github.com/rstudio/blogdown}{\emph{blogdown}}.
\end{itemize}

\hypertarget{grants-and-awards}{%
\subsection{Grants and awards}\label{grants-and-awards}}

\begin{enumerate}
\def\labelenumi{\arabic{enumi}.}
\tightlist
\item
  Grants
\end{enumerate}

\begin{itemize}
\item
  \emph{The Fundamental Research Funds for the Central Universities
  {[}15YBB29{]}} (2015 - 2016) ``Experimental explorations of possible
  world semantics.'' Zhan, L. (¥20,000)
\item
  \emph{The Funds Supporting the Growth of the New Teachers
  {[}FD201530{]}} 2015 - 2016 ``Introduction to psychology and
  introduction to scientific research.'' Zhan, L. (¥7,000)
\item
  \emph{The Fundamental Research Funds for the Central Universities,
  {[}15YJ050003{]}} (2015 - 2016) ``Processing and acquisition of tone
  perception in Mandarin Chinese'' Zhan, L., Shi, F., Gao, L., \& Zhang,
  L. (¥70,000)
\item
  \emph{Cognitive Science Postgraduate Research Grant of Macquarie
  University} (2010 - 2014) ``The interpretation of conditionals in
  natural language'' Zhan, L. (\$10,570)
\item
  \emph{Macquarie University Postgraduate Research Fund} (2013) ``The
  hypothetical property of conditionals in natural language'' Zhan, L.
  (\$4,684)
\end{itemize}

\begin{enumerate}
\def\labelenumi{\arabic{enumi}.}
\setcounter{enumi}{1}
\tightlist
\item
  Awards
\end{enumerate}

\begin{itemize}
\item
  2013.11. 38th Boston University Conference on Language Development,
  Paula Menyuk Travel Award, \$300.
\item
  2013.04. 26th Annual CUNY Sentence Processing Conference, Travel
  Award, \$350.
\end{itemize}

\hypertarget{publications}{%
\subsection{Publications}\label{publications}}

\begin{enumerate}
\def\labelenumi{\arabic{enumi}.}
\tightlist
\item
  Book
\end{enumerate}

\begin{itemize}
\tightlist
\item
  \textbf{Zhan, L.} (2015). \emph{The Interpretation of Conditionals in
  Natural Language}. Saarbrucken, Germany: Lap Lambert Academic
  Publishing.
\end{itemize}

\begin{enumerate}
\def\labelenumi{\arabic{enumi}.}
\setcounter{enumi}{1}
\tightlist
\item
  Periodicals
\end{enumerate}

\begin{itemize}
\item
  Zhou, P., \textbf{Zhan, L.}, \& Ma, H. (2018). Predictive language
  processing in preschool children with Autism Spectrum Disorder: An
  eye-tracking study. \emph{Journal of Psycholinguistic Research}.
  doi:10.1007/s10936-018-9612-5
  \href{https://publications.likan.info/Periodicals/JPsycholinguistRes2018.pdf}{(PDF)}
\item
  \textbf{Zhan, L.} (2018). Using eye movements recorded in the visual
  world paradigm to explore the online processing of spoken language.
  \emph{Journal of Visualized Experiments, 140}, e58086. doi:
  10.3791/58086
  \href{https://publications.likan.info/Periodicals/jove-protocol-58086.pdf}{(PDF)}
\item
  Zhou, P., Ma, W., \textbf{Zhan, L.}, \& Ma, H (2018). Using the visual
  world paradigm to study sentence comprehension in Mandarin-speaking
  children with autism. \emph{Journal of Visualized Experiments, 140},
  e58452. doi: 10.3791/58452
  \href{https://publications.likan.info/Periodicals/jove-protocol-58452.pdf}{(PDF)}
\item
  \textbf{Zhan, L.}, Zhou, P., \& Crain, S. (2018). Using the
  visual-world paradigm to explore the meaning of conditionals in
  natural language. \emph{Language, Cognition and Neuroscience, 33}(8),
  1049-1062. doi: 10.1080/23273798.2018.1448935
  \href{https://publications.likan.info/Periodicals/LangCognNeurosci2018.pdf}{(PDF)}
\item
  \textbf{Zhan, L.} (2018). Scalar and ignorance inferences are both
  computed immediately upon encountering the sentential connective: The
  online processing of sentences with disjunction using the visual world
  paradigm. \emph{Frontiers in Psychology, 9}. doi:
  10.3389/fpsyg.2018.00061
  \href{https://www.frontiersin.org/articles/10.3389/fpsyg.2018.00061/full}{(PDF)}
\item
  Moscati, V., \textbf{Zhan, L.}, \& Zhou, P. (2017). Children's on-line
  processing of epistemic modals. \emph{Journal of Child Language,
  44}(5), 1025-1040. doi: 10.1017/S0305000916000313\\

  \href{https://publications.likan.info/Periodicals/JChildLang2016.pdf}{(PDF)}
\item
  \textbf{Zhan, L.}, Crain, S., \& Zhou, P. (2015). The online
  processing of only if- and even if- conditional statements:
  Implications for mental models. \emph{Journal of Cognitive Psychology,
  26}(7), 367-379. doi: 10.1080/ 20445911.2015.1016527
  \href{https://publications.likan.info/Periodicals/JCognPsychol2015.pdf}{(PDF)}
\item
  Zhou, P., Crain, S., \& \textbf{Zhan, L.} (2014). Grammatical aspect
  and event recognition in children's online sentence comprehension.
  \emph{Cognition, 133}(1), 262-276. doi:
  10.1016/j.cognition.2014.06.018
  \href{http://publications.likan.info/Periodicals/Cognition2014.pdf}{(PDF)}
\item
  Zhou, P., Crain, S., \& \textbf{Zhan, L.} (2012). Sometimes children
  are as good as adults: The pragmatic use of prosody in children's
  on-line sentence processing. \emph{Journal of Memory and Language,
  67}(1), 149-164. doi: 10.1016/j.jml.2012.03.005
  \href{https://publications.likan.info/Periodicals/JMemLang2012.pdf}{(PDF)}
\item
  Zhou, P., Su, Y., Crain, S., Gao, L., \& \textbf{Zhan, L.} (2012).
  Children's use of phonological information in ambiguity resolution: a
  view from Mandarin Chinese. \emph{Journal of Child Language, 39}(04),
  687-730. doi: 10.1017/S0305000911000249
  \href{https://publications.likan.info/Periodicals/JChildLang2012.pdf}{(PDF)}
\end{itemize}

\begin{enumerate}
\def\labelenumi{\arabic{enumi}.}
\setcounter{enumi}{2}
\tightlist
\item
  Book chapters and conference proceedings
\end{enumerate}

\begin{itemize}
\item
  \textbf{Zhan, L.} (2018). Magnetoencephalography (MEG) as a Technique
  for Imaging Brain Function and Dysfunction. In \emph{Top 10
  Contributions on Psychology} (Chapter 4, pp.~1-38). Telangana, India:
  Avid Science
\item
  \textbf{Zhan, L.}, Crain, S., \& Zhou, P. (2013). The anticipatory e
  ects of focus operators: A visual- world paradigm eye-tracking study
  of ``only if'' and ``even if'' conditionals. In N. Goto, K. Otaki, A.
  Sato, \& K. Takita (Eds.), \emph{Proceedings of GLOW in Asia IX 2012}.
  Mie University, Japan.
\end{itemize}

\begin{enumerate}
\def\labelenumi{\arabic{enumi}.}
\setcounter{enumi}{3}
\tightlist
\item
  Conference presentations and invited talks
\end{enumerate}

\begin{itemize}
\item
  \textbf{Zhan, L.} (2018, December). \emph{Sentential Reasoning and
  Sentential Connectives: Conditional, Disjunction, Negation, and
  Modality}. Inivted presentation given at the Workshop of Theoretical
  and Experimental Linguistics, Tsinghua University, Beijing, China.
  \href{https://publications.likan.info/Talks/Sentential_Reasoning_Sentential_Connectives.pdf}{
  (PDF) }
\item
  \textbf{Zhan, L.} (2018, November). \emph{Methods of Cognitive
  Neuroscience: Focus on Language}. Inivted presentation given at the
  Child Cognition Laboratory, Department of Foreign Languages and
  Literatures, Tsinghua University, Beijing, China.
  \href{https://publications.likan.info/Talks/MethodsCognNeurosciLang2018NOV.pdf}{
  (PDF) }
\item
  \textbf{Zhan, L.} (2018, November). \emph{Visual world paradigm: An
  eye-tracking technique to study the real time processing of spoken
  Language}. Inivted presentation given at the Center for Studies of
  Chinese as a Second Language, Beijing Language and Culture University,
  Beijing, China.
  \href{https://publications.likan.info/Talks/Visual_World_Paradigm.pdf}{
  (PDF) }
\item
  \textbf{Zhan, L.} (2018, October). \emph{Experimental Builder: A
  What-You-See-Is-What-You-Get tool to build experiment scripts}.
  Inivted presentation given at the Center for Studies of Chinese as a
  Second Language, Beijing Language and Culture University, Beijing,
  China.
  \href{https://publications.likan.info/Eyelink_Experiment_Builder_Training_Materials/}{
  (Example materials) }
\item
  \textbf{Zhan, L.} (2017, September). \emph{Scalar implicature and
  ignorance inference are both locally computed: Evidence from the
  online processing of disjunctions using the visual world paradigm}.
  Paper presented at the Second High-level Forum on Cognitive
  Linguistics, University of International Business and Economics,
  Beijing, China.
  \href{https://publications.likan.info/Talks/ZhanL2017UIBE.pdf}{ (PDF)
  }
\item
  \textbf{Zhan, L.}, Crain, S., \& Zhou, P. (2013, November).
  \emph{Going beyond the information that is perceived: The hypothetical
  property of if-conditionals in Mandarin Chinese}. Paper session
  presented at the Second International Conference on Psycholinguistics
  in China, Fuzhou, China.
\item
  Moscati, V., \textbf{Zhan, L.}, \& Zhou, P. (2013, November).
  \emph{Reasoning on possibilities: An eye tracking study on modal
  knowledge}. Poster session presented at the 38th Annual Boston
  University Conference on Language Development, Boston University, MA.
\item
  Zhou, P., Crain, S., \& \textbf{Zhan, L.} (2013, November).
  \emph{Anticipatory eye movements in children's processing of
  grammatical aspect}. Poster session presented at the 38th Annual
  Boston University Conference on Language Development, Boston
  University, MA.
\item
  \textbf{Zhan, L.}, Crain, S., \& Zhou, P. (2013, March). \emph{The
  hypothetical property of ``if''-statements: A visual- world paradigm
  eye-tracking study}. Poster session presented at CUNY2013: The 26th
  annual CUNY Sentence Processing Conference, Columbia, SC.
\item
  \textbf{Zhan, L.}, Crain, S., \& Zhou, P. (2012, July). \emph{The
  interpretation of conditionals}. Paper presented at the 7th
  International Conference on Thinking (ICT2012), Birkbeck/UCL, London,
  UK.
\item
  Zhou, P., Crain, S., \& \textbf{Zhan, L.} (2012, March).
  \emph{Children's pragmatic use of prosody in sentence processing}.
  Poster session presented at the 35th Generative Linguistics in the Old
  World (GLOW) Workshop: Production and Perception of
  Prosodically-Encoded Information Structure, University of Potsdam,
  Potsdam, Germany.
\item
  \textbf{Zhan, L.}, Crain, S., \& Khlentzos, D. (2011, August).
  \emph{The basic semantics of conditionals in natural language}. Paper
  presented at the Harvard-Australia Workshop on Language, Learning and
  Logic, Macquarie University, Sydney, Australia.
\item
  Zhou, P., Crain, S., Gao, L., \& \textbf{Zhan, L.} (2010, September).
  \emph{The role of prosody in children's focus identification}. Paper
  presented at the Generative Approaches to Language Acquisition - North
  America 4 (GALANA-4), Toronto, Canada.
\item
  Zhou, P., Su, Y., Crain, S., Gao, L., \& \textbf{Zhan, L.} (2010,
  August). \emph{Children's use of prosodic information in ambiguity
  resolution}. Paper presented at the 8th Conference of Generative
  Linguistics in the Old World Asia (GLOW-in-Asia 8), Beijing, China.
\end{itemize}

\end{document}
